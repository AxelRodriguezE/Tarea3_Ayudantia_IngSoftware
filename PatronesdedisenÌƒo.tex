\documentclass[a4paper,11pt]{report}
%
%--------------------   start of the 'preamble'
%
\usepackage{graphicx,amssymb,amstext,amsmath}
\usepackage{enumerate}
\usepackage{hyperref}
\usepackage[spanish]{babel}
\usepackage[utf8]{inputenc}
\hypersetup{
    colorlinks,
    citecolor=black,
    filecolor=black,
    linkcolor=black,
    urlcolor=black
}
%
%%    homebrew commands -- to save typing
\newcommand\etc{\textsl{etc}}
\newcommand\eg{\textsl{eg.}\ }
\newcommand\etal{\textsl{et al.}}
\newcommand\Quote[1]{\lq\textsl{#1}\rq}
\newcommand\fr[2]{{\textstyle\frac{#1}{#2}}}
\newcommand\miktex{\textsl{MikTeX}}
\newcommand\comp{\textsl{The Companion}}
\newcommand\nss{\textsl{Not so Short}}
%
%---------------------   end of the 'preamble'
%
\begin{document}
%-----------------------------------------------------------
\title{Patrones de Diseño}
\author{Axel Rodríguez Espinoa\\
        Nicolas Oyarzún Hernandez\\
        Pedro Salas Vergara}
\maketitle

%-----------------------------------------------------------
\title {\textbf{PATRONES DE DISEÑO DE SOFTWARE}}\\

\title{\textbf{Introducción}}\\

El diseño es un modelo del sistema, realizado con una serie de principios y técnicas, que permite describir el sistema con el suficiente detalle como para ser implementado. Pero los principios y reglas no son suficientes, en el contexto de diseño podemos observar que los buenos ingenieros tienen esquemas y estructuras de solución que usan numerosas veces en función del contexto del problema. Este es el sentido cabal de la expresión "tener un mente bien amueblada", y no el significado de tener una notable inteligencia. Estos esquemas y estructuras son conceptos reusables y nos permiten no reinventar la rueda. Un buen ingeniero reutiliza un esquema de solución ante problemas similares.\\

\title{\textbf{Antecedentes}}\\

El concepto de "patrón de diseño" que tenemos en Ingeniería del Software se ha tomado prestado de la arquitectura. En 1977 se publica el libro "A Pattern Language: Towns/Building/Construction", de Christopher Alexander, Sara Ishikawa, Murray Silverstein, Max Jacobson, Ingrid Fiksdahl-King y Shlomo Angel, Oxford University Press. Contiene numerosos patrones con una notación específica de Alexander.\\

\title{\textbf{Concepto de patron de diseño}}\\



\title{\textbf{Ventajas de los patrones de diseño}}\\



\title{\textbf{Clasificación de patrones}}\\



\title{\textbf{EL patron de diseño que vamos a ejemplificar}}\\






\begin{itemize}
    \item{Modelo}
    \item{Vista}
    \item{Controlador}
\end{itemize}

\begin{figure}[!ht]
\begin{center}

\caption{MVC aplicado en un software}
\end{center}
\end{figure}


 \newpage
La capa de modelo mapea los datos de la aplicación; en está es en la que almacenaremos datos para buscar/inserción/eliminación y/o modificación.
La capa del controlador es muy importante para este patrón de diseño, ya que es la
que maneja la lógica con la que se manipulan los datos (Es decir se relaciona íntimamente con
la capa modelo) y a su vez interactúa con la capa vista, mostrando al cliente la representación de los datos según sus peticiones.\\

\title {\textbf{ EJEMPLO DE MVC}}\\

\underline{Problema:} Realizar la venta de un producto en una tienda.

\underline{Aplicación del patrón MVC al problema de la tienda:}

En este problema podemos mapear las 3 capas de MVC a la tienda, teniendo lo siguiente:

\begin{itemize}
    \item{Modelo:} Libro de ventas, productos de la tienda, etc. 
    \item{Vista:} Estantes con productos, precio de los productos.
    \item{Controlador:} Vendedor realizando ventas, vendedor realizando boletas.
\end{itemize}

El cliente solicitará al vendedor que le venda un producto (Una frucola grande),
a lo que el vendedor le dirá que esta tiene un precio de \$350 y le pedirá el dinero especificado al cliente.\\
El cliente pagará su producto y el vendedor se lo entregará al cliente con su boleta correspondiente.\\
Cada petición que hace el cliente al vendedor es una interacción con
un controlador, esté a su vez entrega información sobre productos y solicita información al cliente mediante vistas.\\
Se producirán cambios al modelo (ej: Cuando el vendedor le entrega
un producto o la boleta al cliente) y estos son un efecto colateral a las peticiones realizadas al controlador.

\end{document}